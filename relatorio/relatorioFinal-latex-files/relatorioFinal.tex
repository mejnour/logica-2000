\documentclass[11pt, a4paper]{article}

	% Linguagem e Escrita
\usepackage[brazil]{babel}
\usepackage[utf8]{inputenc}
\usepackage[T1]{fontenc}
\usepackage{indentfirst}

	% Imagens
\usepackage{tikz}

  % Documentos
\usepackage{pdfpages}

	% Layout
\usepackage[some, center]{background}
\usepackage{geometry}
	\geometry{paperwidth = 210mm,%
		      paperheight = 297mm,%
	    	  layout = a4paper,%s
	      	textheight = 24cm%
	      	}

	% Codigo
\usepackage{listings}

\begin{document}

		% --- Folha de Rosto ---

	\begin{titlepage}

		\SetBgContents{\includegraphics{ufpb.png}}
		\SetBgScale{.8}
		\SetBgAngle{0}
		\SetBgOpacity{0.1}
		\BgThispage

    	\begin{center}

	        {\sc \Large Universidade Federal da Paraíba - \textbf{U.F.P.B.}; \\
    	        Centro de Informática - \textbf{C.I.}; \\
        	    Departamento de Ciência da Computação - \textbf{D.C.C.}, Campus \textbf{V}; \\
            	Disciplina: Linguagens Formais e Autômatos; \\
	            Período: 2017.2 \\}

        	\vspace{1cm}

    	    {\sc Professor: Andrei Formiga}

	        \vspace{4cm}

        	{\LARGE \sc Platafotma Web de Estudos \\ {\Large Em Disciplinas Correlatas a Lógica} \\}

					\vspace{3cm}

	        \begin{flushleft}

            	{\sf \sc \large Relatório: \bf{Final} \\}

        	\end{flushleft}

    	    \vspace{2cm}

	        \begin{flushleft}

            	{\sf \sc {\bf Aluno:} José Arnaldo de Assis Pina Neto 									\footnote{Matrícula: 2016064100}\\
																		Juliano Nunes dos Santos
																		\footnote{Matrícula: 2016001002}\\
																		Paulo Henrique Bezerra Matias
																		\footnote{Matrícula: 2016057509}\\
																		Renan Ribeiro Lage
																		\footnote{Matrícula: 2016057509}}

        	\end{flushleft}

    	    \vspace{1.5cm}

	        \sc{João Pessoa,} \\
        	21 de junho de 2018

    	\end{center}
	\end{titlepage}

		% ----------------------

	\pagebreak

	% \mainmatter

	\section{Introdução}

		Dentre as disciplinas correlatas à lógica em cursos de graduação para computação, algumas frequentemente apresentam um desafio aos novos ingressantes do ensino superior. Isto, pois, deparam-se com um modo abstrato de tratar questões antes introduzidas através de código em disciplinas de introdução à programação. Não somente em Lógica Aplicada à Computação, mas também em Teoria da Computação e em Linguagens Formais e Autômatos, existe uma curva crescente de abstração e um conhecimento acumulativo que precisa ser constantemente revisitado.\\

		Vindo ao auxílio daqueles que sentem a necessidade anteriormente citada, este projeto visou criar uma plataforma online que serviria como banco de questões, podendo o usuário, ainda, visitar trechos de teoria, trocar ideias com colegas em relação à questões e interagir sobre os temas propostos.\\

		Pensou-se, aos usuários, a possibilidade de se oferecer um serviço que unisse rede social e estudos, de maneira dinâmica, onde o usuário ganharia pontos por questões respondidas. O professor também poderia criar uma conta em que tivesse acesso ao desenvolvimento dos alunos. Este relatório descreve esta empreitada em detalhes: êxitos e falhas.

	\section{Resultados}

		O projeto foi desenvolvido usando HTML5, CSS3 e JavaScript. Encontra-se hospedado neste repositório\footnote{} no GitHub.

	\section{Conclusão}

		Durante o período 2017.2 da UFPB, na disciplina de Linguagens Formais e Autômatos, ministrada pelo Professor Andrei Formiga, foi possível aos alunos terem um primeiro contato com o que há de ponta no desenvolvimento científico relativo à Computação.\\

		Desde uma revisão dos conteúdos baselares, passando por complexidade de algoritmos, entre outros tópicos, ficou claro aos atendentes da disciplina que o universo da computação era muito mais amplo e profundo do que descreviam os pequenos mapas simbólicos da situação dados pelos professores até então.\\

		Assim, a possibilidade de desenvolver um projeto aberto, sem finalidade previamente definida pelo professor, foi uma oportunidade única de explorar ferramentas não usuais nas tradicionais disciplinas do curso, contribuíndo com a experiência que é sanar uma curiosidade e o treinamento profissional auto-imposto.


\end{document}
